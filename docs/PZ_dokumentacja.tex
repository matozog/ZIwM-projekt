\documentclass[12pt]{article}
\usepackage{polski}
\usepackage[utf8]{inputenc}
\usepackage{fullpage}
\usepackage{tabto}
\usepackage{graphicx} 
\usepackage{float}
\usepackage{caption}


\linespread{1.3}
\begin{document}
%---------------------------------------------------------
%					Strona Tytułowa
%---------------------------------------------------------
\begin{titlepage}
%-----------------------Tytuł-----------------------------
\newcommand{\LINE}{\rule{\linewidth}{0.7mm}}
\center
\LINE \\[0.5cm]
\Large\textsc{Komputerowe wspomaganie diagnozowania nowotworów piersi z wykorzystaniem algorytmów minimalno-odległościowych}\\ [5mm]
\normalsize\textsc{Zastosowanie Informatyki w Medycynie}  \\[0.5cm]
\LINE \\[3cm]
%----------------------Nazwiska---------------------------
\begin{minipage}{0.5\textwidth}
\begin{flushleft} \large
\emph{Autorzy:}
		\\Mateusz Ożóg %226125
		\\Grzegorz Milaszkiewicz %226110
\end{flushleft}
\end{minipage}
~
\begin{minipage}{0.45\textwidth}
\begin{flushright} \large
\emph{Prowadzący:} \\
mgr inż. Jakub Klikowski
\end{flushright}
\end{minipage}\\[2cm]
%----------------------Stopka-----------------------------
\vfill
\center Wrocław 2019
\end{titlepage}

%---------------------------------------------------------
%					Spis treści
%---------------------------------------------------------
\renewcommand{\contentsname}{Spis treści}
\tableofcontents
\newpage

%---------------------------------------------------------
%					Część pierwsza
%---------------------------------------------------------
\section{Charakterystyka analizowanego problemu}
Charakterystyka analizowanego problemu (jako zadania rozpoznawania)

%---------------------------------------------------------
%					Część druga
%---------------------------------------------------------
\section{Opis stosowanych algorytmów}


%---------------------------------------------------------
%					Część Trzecia
%---------------------------------------------------------
\section{Informacja o środowisku implementacyjnym}



%	\begin{figure}[H]
%		\centering
%		\includegraphics[scale=0.4]{obsluga_konta.png}
%		\caption{Diagram przypadków użycia dla modułu "Obsługa konta"}
%	\end{figure}

%--------------------Moduł obsługi konta---------------------------
\begin{table}[H]
	\begin{tabular}{|p{0.3\linewidth}|p{0.64\linewidth}|}%{|l|l|}
	\hline
	Id wymagania 	& 1 				\\ \hline
	Nazwa			& Logowanie \\ \hline
	\end{tabular}
\captionof{table}{Karta wymagania dla przypadku użycia Logowanie} 
\end{table}



%---------------------------------------------------------
%					Część czwarta
%---------------------------------------------------------
\section{Opis badań eksperymentalnych}


%---------------------------------------------------------
%					Część piąta
%---------------------------------------------------------
\section{Wyniki}
Wyniki (w formie graficznej, tabeli, itp.)



%---------------------------------------------------------
%					Część szósta
%---------------------------------------------------------

\section{Dyskusja wyników i wnioski wynikające z badań.}

\end{document}


