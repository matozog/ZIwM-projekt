\documentclass[12pt]{article}
\usepackage{polski}
\usepackage[utf8]{inputenc}
\usepackage{fullpage}
\usepackage{tabto}
\usepackage{graphicx} 
\usepackage{float}
\usepackage{caption}


\linespread{1.3}
\begin{document}
%---------------------------------------------------------
%					Strona Tytułowa
%---------------------------------------------------------
\begin{titlepage}
%-----------------------Tytuł-----------------------------
\newcommand{\LINE}{\rule{\linewidth}{0.7mm}}
\center
\LINE \\[0.5cm]
\Huge\textsc{RAK CYCKÓW}\\ [5mm]
\normalsize\textsc{ZIwM}  \\[0.5cm]
\LINE \\[3cm]
%----------------------Nazwiska---------------------------
\begin{minipage}{0.5\textwidth}
\begin{flushleft} \large
\emph{Autorzy:}
		\\Mateusz Ożóg %226125
		\\Grzegorz Milaszkiewicz %226110
\end{flushleft}
\end{minipage}
~
\begin{minipage}{0.45\textwidth}
\begin{flushright} \large
\emph{Prowadzący:} \\
Klikowski
\end{flushright}
\end{minipage}\\[2cm]
%----------------------Stopka-----------------------------
\vfill
\center Wrocław 2019
\end{titlepage}

%---------------------------------------------------------
%					Spis treści
%---------------------------------------------------------
\renewcommand{\contentsname}{Spis treści}
\tableofcontents
\newpage

%---------------------------------------------------------
%					Część pierwsza
%---------------------------------------------------------
\section{Czynniki sterujące projektem}
\subsection{Cele projektu} 


\begin{itemize}
\item \textbf{Nauczyciele} - użytkownicy aplikacji webowej. Mają największy udział w użytkowaniu systemu. Styczność z systemem będzie odbywała się codziennie poprzez prowadzenie dziennika elektronicznego. 
\item \textbf{Managerowie} - podobnie jak nauczyciele są użytkownikami aplikacji webowej. Ich rola interakcji z systemem będzie polegała na zarządzaniu grupami oraz kontami uczniów i~nauczycieli.
\item \textbf{Uczniowie} - użytkownicy obydwu aplikacji. Dla tej grupy użytkowników system będzie pełnił w głównej mierze rolę informacyjną.
\end{itemize}


%---------------------------------------------------------
%					Część druga
%---------------------------------------------------------
\clearpage
\section{Ograniczenia projektu}
\subsection{Ograniczenia wynikające z natury i okoliczności projektu}
\textbf{Ograniczenia wynikające z harmonogramu}\\
Projekt ma zostać ukończony do końca bieżącego semestru akademickiego. Aplikacja webowa oraz mobilna powinna zostać ukończona do dnia 13 czerwca 2018r.\\

\noindent\textbf{Ograniczenia wynikające ze środowiska pracy}\\
Aplikacja mobilna musi działać na urządzeniach z systemem operacyjnym Android w wersji 5.0 lub nowszej. Aplikacja webowa ma działać na przeglądarkach Opera, Chrome i Mozilla Firefox.
\subsection{Konwencje nazewnicze i definicje}
\textbf{Aplikacja webowa} - aplikacja internetowa; program komputerowy, który pracuje na serwerze i komunikuje się z użytkownikiem poprzez sieć komputerową.\\[0.3cm]
\textbf{Aplikacja mobilna} - program komputerowy działający na urządzeniach przenośnych takich jak smartfony czy tablety.\\[0.3cm]
\textbf{Serwer} - program świadczący usługi na rzecz innych programów, zazwyczaj korzystający z~innych komputerów połączonych w sieć komputerową.\\[0.3cm]
\textbf{Program komputerowy} - aplikacja napisana przez programistę wykonująca określone zadania.
\subsection{Fakty i założenia powiązane z projektem}
System będzie przeznaczony dla osób w różnym wieku, od uczniów szkoły podstawowej, którzy dodatkowo będą uczestniczyć w zajęciach szkoły językowej, aż po najstarszych pracowników szkoły językowej, którzy będą prowadzić zajęcia. Aplikacja mobilna z której będą korzystać tylko uczniowie ma posiadać nowoczesny i intuicyjny interfejs. Aplikacja webowa ma zostać przystosowana dla każdego użytkownika. Interfejs powinien być prosty i przejrzysty. \\[0.5cm]
Do zarządzania zadaniami w grupie zostanie wykorzystana webowa aplikacja \textit{Trello}. Dzięki niej lider grupy będzie rozdysponowywał nowe zadania członkom grupy oraz nadzorował postęp prac obecnych zadań. Do komunikacji wewnątrz grupy zostanie użyty \textit{Slack}. Zostaną na nim stworzone trzy grupy. Pierwsza z nich będzie będzie główną grupą gdzie obecni będą wszyscy członkowie. Do drugiej grupy zostaną dodani członkowie odpowiedzialni za aplikację webową. Trzecia grupa będzie analogiczna do drugiej z tą różnicą, że będą w niej obecni członkowie odpowiedzialni za aplikacje mobilną.

%---------------------------------------------------------
%					Część Trzecia
%---------------------------------------------------------
\clearpage
\section{Wymagania funkcjonalne}
\subsection{Zakres produktu}
Zakres produktu został przedstawiony poniżej za pomocą diagramów przypadków użycia. Funkcjonalności każdego modułu zostaną szerzej opisane w dalszej części dokumentacji. Moduły przeznaczone dla aplikacji webowej prezentują się następująco:



%	\begin{figure}[H]
%		\centering
%		\includegraphics[scale=0.4]{obsluga_konta.png}
%		\caption{Diagram przypadków użycia dla modułu "Obsługa konta"}
%	\end{figure}


	
	
\clearpage
\subsection{Wymagania funkcjonalne i dotyczące danych}
%--------------------------Moduły aplikacji webowej---------------------------
Szczegółowy opis przypadków użycia dla aplikacji webowej został przedstawiony poniżej.
%--------------------Moduł obsługi konta---------------------------
\begin{table}[H]
	\begin{tabular}{|p{0.3\linewidth}|p{0.64\linewidth}|}%{|l|l|}
	\hline
	Id wymagania 	& 1 				\\ \hline
	Nazwa			& Logowanie \\ \hline
	\end{tabular}
\captionof{table}{Karta wymagania dla przypadku użycia Logowanie} 
\end{table}



%---------------------------------------------------------
%					Część czwarta
%---------------------------------------------------------
\newpage
\section{Wymagania poza funkcjonalne}
\subsection{Wymagania estetyczne}



\end{document}


